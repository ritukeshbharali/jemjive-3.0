\documentclass[a4paper]{article}

\usepackage{listings}
\usepackage{palatino}
\usepackage{parskip}
\usepackage{color}
\usepackage{url}

\lstset{language={},backgroundcolor=\color[gray]{0.95},
  basicstyle=\ttfamily}


%========================================================================
% Definitions
%========================================================================


\newcommand{\Code}[1]{\texttt{#1}}


%========================================================================
% Document
%========================================================================


\title {Jem/Jive 2.1\\[1ex]
        Release Notes}
\author{Dynaflow Research Group}
\date  {11-6-2015}


\begin{document}

\maketitle

\section{Overview}

Version 2.1 of Jem/Jive is a minor release with some new features,
improvements and bug fixes. This version should be source compatible with
version 2.0.

The following list provides an overview of the new features and
improvements. More detailed information can be found in the following
sections.

\begin{itemize}

  \item Support for quadrilateral elements with 12 nodes and cubic shape
        functions. Use the class \Code{Quad12} to create a \Code{Shape}
        object for calculating the shape functions.

  \item Support for reading and writing HDF5 files.

  \item Extension of the \Code{Properties} class with operators for
        getting and setting properties.

  \item Switch from headers provided by the standard C library to
        the equivalent headers provided by the standard C++ library.

  \item Some code changes to better conform to the recent C++ standard
        and the latest C++ compilers.

  \item Support for creating static libraries using the Makefiles
        provided by Jem and Jive.

  \item Support for the \Code{clang} compiler.

  \item Better support for recent versions of MacOS~X.

  \item Bug fix in the \Code{ArclenModule} so that it properly handles
        linear constraints.

  \item Various small Windows-specific bug fixes.

  \item Some internal changes and code re-organisations to increase code
        re-use.

\end{itemize}

%------------------------------------------------------------------------

\section{Support for HDF5 files}

The new Jem package \Code{hdf5} provides an interface for creating,
modifying and reading HDF5 data files. The interface automatically
translates Jem objects into equivalent HDF5 objects. See
\url{www.hdfgroup.org/HDF5/} for more information about HDF5.

%------------------------------------------------------------------------

\section{Extension of the \Code{Properties} class}

The \Code{Properties} class from Jem implements the \Code{[]} (subscript)
operator that can be used to both get and set properties. Here is an
example:

\begin{lstlisting}
  Properties  p;
  String      s;
  double      x;
  int         i;

  s = "hello";
  x = 0.0;
  i = 1;

  p["first"]  = s;
  p["second"] = x;
  p["third"]  = i;

  s = p["first"];
  x = p["second"];
  i = p["third"];
\end{lstlisting}

The operator calls are automatically translated to the equivalent calls
the \Code{get} and \Code{set} member functions.

%------------------------------------------------------------------------

\section{Create static libraries}

Use the \Code{lib.mk} Makefile to build a static library instead of an
executable. Here is an example Makefile:

\begin{lstlisting}
  library = femutil

  include $(JIVEDIR)/makefiles/packages/fem.mk
  include $(JIVEDIR)/makefiles/lib.mk
\end{lstlisting}

\end{document}

