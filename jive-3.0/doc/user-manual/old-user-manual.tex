%%%%%%%%%%%%%%%%%%%%%%%%%%%%%%%%%%%%%%%%%%%%%%%%%%%%%%%%%%%%%%%%%%%%%%%%%%%

\documentclass[a4paper,12pt,english]{habanera-shortmanual}

\usepackage{epsfig}


\newcommand{\Code}[1]{\texttt{#1}}



\Title{Jive user manual}

\Subtitle{}
\nocontents


%%%%%%%%%%%%%%%%%%%%%%%%%%%%%%%%%%%%%%%%%%%%%%%%%%%%%%%%%%%%%%%%%%%%%%%%%%%

\begin{document}

\noindent
Jive is an object oriented toolkit, written in C++, that can be used
to solve partial differential equations. It consists of a set of
classes and functions for transforming a partial differential equation
into a sparse system of equations; for solving such a system of
equations; and for computing quantities derived from the solution.

Jive is built on top of the general purpose toolkit Jem. This means
that Jive will run on all platforms supported by Jem. It also means
that users of Jive will have to become familiar with a subset of Jem.

\BlankLine
Like Jem, Jive bundles groups of related classes and functions into
modular units called \emph{packages}. This version of Jive contains
the following packages:
\begin{itemize}

\item \Code{algebra} : contains a set of classes that implement common
  matrix/vector operations. It also provides a framework for writing
  generic algorithms involving matrices and vectors.

\item \Code{fem}: exports various classes that can be used to develop
  finite element applications. In particular, it exports classes for
  assembling global matrices and a global vectors from sets of element
  matrices and vectors.

\item \Code{geom}: provides a set of classes that can be used to
  compute the interpolation functions of basic shapes such as lines,
  triangles, quadrilaterals, tetrahedra, hexahedra, wedges, and
  multi-dimensional prisms. These classes also provide support for
  computing the spatial derivatives of the interpolation polynomials
  and for evaluating surface and volume integrals.

\item \Code{io}: contains classes for reading and writing meshes and
  other data structures. All data are written to and read from
  xml-formatted text files. This means that one can directly view the
  contents of an output file. What is more, one can easily create an
  input file, either by hand or automatically.

\item \Code{solver} : provides support for solving linear systems of
  equations. It comprises two classes that implement an iterative
  solution algorithm, and one class that implements a direct
  algorithm.

\item \Code{util} : exports various classes and functions for storing
  and manipulating frequently used data structures such as
  unstructured grids and finite element meshes. This package is used
  by all other packages in Jive.

\end{itemize}

\BlankLine
Writing a program is Jive is similar to writing a program with
Jem. Consult the Jem user manual to find out how to use a package, how
to compile and link a program, and how to write a portable make file.

\BlankLine
The capabilities of this version of Jive are demonstrated by three
sample program that can be found in the directory
\Code{\$JIVEDIR/examples}, where \Code{\$JIVEDIR} denotes the
directory in which Jive has been installed. The first sample program
-- located in the subdirectory \Code{poisson-fdm} -- solves a simple
Poisson problem by means of the finite difference method. The second
program -- located in \Code{poisson-fem} -- solves the same problem by
means of the finite element method. The third program -- located in
\Code{space-time} -- solves a time-dependent diffusion problem by
means of the finite element method. The interesting thing about this
program is that the elements are defined in the space-time domain, and
not, as is usual, in spatial domain alone.


\end{document}

