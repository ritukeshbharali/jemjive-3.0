%%%%%%%%%%%%%%%%%%%%%%%%%%%%%%%%%%%%%%%%%%%%%%%%%%%%%%%%%%%%%%%%%%%%%%%%%
% Preamble
%%%%%%%%%%%%%%%%%%%%%%%%%%%%%%%%%%%%%%%%%%%%%%%%%%%%%%%%%%%%%%%%%%%%%%%%%

\documentclass[a4paper,11pt]{article}

\usepackage{array}
\usepackage{longtable}
\usepackage{palatino}

\pagestyle{plain}


\topmargin        0cm
\headheight       0cm
\headsep          0cm
\textheight      24cm
\textwidth       15cm
\evensidemargin   0cm
\oddsidemargin    0cm


%%%%%%%%%%%%%%%%%%%%%%%%%%%%%%%%%%%%%%%%%%%%%%%%%%%%%%%%%%%%%%%%%%%%%%%%%
% Macros
%%%%%%%%%%%%%%%%%%%%%%%%%%%%%%%%%%%%%%%%%%%%%%%%%%%%%%%%%%%%%%%%%%%%%%%%%


\newcommand{\BlankLine}{\vspace{1.5ex} \noindent}
\newcommand{\Code}[1]{\texttt{#1}}
\newcommand{\PBS}[1]{\let\temp=\\#1\let\\=\temp}


%%%%%%%%%%%%%%%%%%%%%%%%%%%%%%%%%%%%%%%%%%%%%%%%%%%%%%%%%%%%%%%%%%%%%%%%%
% Title
%%%%%%%%%%%%%%%%%%%%%%%%%%%%%%%%%%%%%%%%%%%%%%%%%%%%%%%%%%%%%%%%%%%%%%%%%


\title{ Jive 1.0 overview }

\author{ Habanera }

\date{ \today }



%%%%%%%%%%%%%%%%%%%%%%%%%%%%%%%%%%%%%%%%%%%%%%%%%%%%%%%%%%%%%%%%%%%%%%%%%
% document
%%%%%%%%%%%%%%%%%%%%%%%%%%%%%%%%%%%%%%%%%%%%%%%%%%%%%%%%%%%%%%%%%%%%%%%%%


\begin{document}

\maketitle

%------------------------------------------------------------------------

\section{Introduction}

Jive~1.0 is the successor of Jive~0.9. It is much more flexible than
its predecessor and it offers many new features that will make it
easier to develop advanced numerical applications. It also fixes some
design flaws that have been exposed by the users of the preceding
versions of Jive.

Although the overall structure of Jive~1.0 largely resembles the one
of Jive~0.9, programs written for Jive~0.9 will not compile with
Jive~1.0. This means that the users of Jive~0.9 will have to modify
their programs in order to use Jive~1.0. Habanera is aware that this
is an unfortunate situation, and will therefore help to port Jive~0.9
programs whenever that is needed. Habanera will also make a commitment
to avoid major incompatibilities between Jive~1.0 and the future
versions of Jive that will be released in the next few years.

The remainder of this document provides an overview of the features
that are new in Jive~1.0; it lists all major new classes in Jive~1.0;
and it provides some help for porting Jive programs from version~0.9
to version~1.0.


%------------------------------------------------------------------------

\section{New features}

\begin{itemize}

\item A set of classes for storing and manipulating many small objects
  named \emph{items}. Jive~1.0 provides support for three types of
  items: points, groups, and members. A point is an item that
  represents a location in an $n$-dimensional space; a group is an
  item that represents a collection of other items; and a member is an
  item that represents a part of another item part of an item. A
  finite element, for instance, can be represented by a group of
  points, and an integration point of an element can be represented by
  a member (do not worry if this sounds confusing; a more elaborate
  explanation of member items is given in the next section). Note that
  you are not limited to the above three types of items; you can
  define new types of items by implementing a class that is derived
  from the class \Code{jive::\-util\-::ItemSet}.

\item Classes for storing (large) sets of data that are related to
  items such as nodes and elements. Jive~1.0 also provides facilities
  for reading/writing these data sets to/from a file.

\item Classes for managing degrees of freedom. These classes make it
  possible to add and remove degrees of freedom during program
  execution.

\item Support for removing items -- such as elements and nodes -- from
  the data structures in which they are stored. All data related to
  these items will be automatically removed too.

\item Support for generic parametric shapes in addition to
  isoparametric shapes. It is now possible to wrap a parametric shape
  around a shape object that evaluates its shape functions and their
  gradients in a standard, local coordinate system. The parametric
  shape will transform the gradients from the local coordinate system
  to the global coordinate system.

\item Support for mapping an arbitrary point in the global coordinate
  system to the corresponding point in the local coordinate system of
  a shape. This works for all parametric shapes, including curved
  shapes and boundary shapes (lines in 2-D, surfaces in 3-D).

\item Classes for building sparse matrices with an arbitrary non-zero
  pattern. In contrast to the previous version, version~1.0 of Jive
  enables one to assemble a sparse matrix without having to determine
  its structure in advance.

\item Improved support for storing, adding and deleting linear
  constraints. There is now a unified class that can be used for all
  sorts of linear constraints.

\item Improved support for configuring objects, such as solvers, from
  a configuration file (or some other configuration source). Jive~1.0
  also provides a set of \emph{factory classes} for creating objects
  of which the type has been specified in a configuration file.

\item More thorough runtime checking for programming errors. Most
  classes in Jive~1.0 will check that the arguments passed to their
  member functions are acceptable, and will throw an exception if that
  is not the case.

\item A framework for implementing numerical models. This framework
  makes it possible to treat a model as a black box that performs
  certain actions -- such as the assembly of a matrix -- when given a
  command. The framework facilitates the use of independently
  developed models within one program. It is located in a new Jive
  package named \Code{model}.

\item A framework for implementing numerical programs. This framework
  is based on the idea that a program consists of an assembly of
  \emph{modules} that perform specific tasks. For instance, one module
  will read the input data, another module will assemble and solve a
  system of equations, and a third module will write the output data.
  This design should facilitate the re-use of code in different
  programs. All components that make up this framework are located in
  the Jive package \Code{app}.

\item Many performance improvements, especially in the package
  \Code{geom} of Jive and the package \Code{io} of Jem.

\end{itemize}


%------------------------------------------------------------------------

\section{New classes}

%........................................................................

\subsection{New classes in the \Code{util} package}

\begin{itemize}

\item \Code{ItemSet} -- represents a (large) set of small objects
  named \emph{items}. Items are identified by both an ID and an
  index. The ID of an item is an arbitrary integer number, while the
  index of an item always ranges from zero to the total number of
  items minus one.

\item \Code{ItemGroup} -- represents a selection of items in an
  \Code{ItemSet}. An \Code{ItemGroup} essentially comprises a pointer
  to an \Code{ItemSet} and an array of item indices.

\item \Code{PointSet} -- an item set in which each item represents a
  point in an $n$-dimensional space.

\item \Code{GroupSet} -- an item set in which each item represents a
  group of other items. A \Code{GroupSet} can be viewed as an array of
  \Code{ItemGroup} objects that has been optimized for large numbers
  of groups. A \Code{GroupSet} also provides some functions that are
  not available when using arrays of \Code{ItemGroup} objects. For
  instance, a \Code{GroupSet} provides functions for getting the
  indices of the items in multiple groups and for obtaining the
  maximum group size. These types of operations are not supported by
  an array of \Code{ItemGroup} objects.

\item \Code{MemberSet} -- an item set in which each item represents a
  part of another item. A member comprises two indices: one that
  refers to another item, and one that refers to a part of that item.
  It is up to the user to assign a meaning to the second index. For
  instance, an integration point within an element can be represented
  by a member item that consists of the element index and the index of
  the integration point within that element.

\item \Code{ItemsWrapper} -- a base class for implementing `localized'
  item set wrapper classes that provides meaningful names for the
  member functions of the wrapped item set. For instance, the wrapper
  class \Code{ElementSet} in the \Code{fem} package provides logical
  names for the member functions of the \Code{GroupSet} class that is
  used to store the element connectivities. This means that one can
  get the node indices of an element by invoking the member function
  \Code{get\-Element\-Nodes} on an \Code{ElementSet} instead of the
  member function \Code{get\-Group\-Members} of the underlying
  \Code{GroupSet}.

\item \Code{GroupWrapper} -- a base class for implementing `localized'
  wrapper classes for \Code{ItemGroup} objects. For instance, the
  wrapper class \Code{Element\-Group} in the \Code{fem} package wraps
  around an \Code{ItemGroup} object representing a selection of
  elements.

\item \Code{Table} -- contains double precision data that are related
  to a set of items. A \Code{Table} object can be viewed as a (sparse)
  matrix in which each row corresponds with an item and each column
  with a different type of data. For instance, a 3-D flow field can be
  represented by a \Code{Table} with three columns that contain the
  flow velocities in the $x$, $y$ and $z$-directions, respectively.
  Each row in this table would then contain the three flow velocities
  for an item such as a point in a grid. \Code{Table} objects can be
  read/written to/from a file and can also be converted to vectors by
  means of a \Code{DofSpace} object (see below).

\item \Code{Database} -- a kind of table in which the table entries
  are arrays of double precision or integer numbers. Since the sizes
  of these array can vary dynamically, the \Code{Database} class
  offers a very flexible way to store data that are related to a set
  of items.

\item \Code{DofSpace} -- associates degrees of freedom with a set of
  items. That is, a \Code{DofSpace} object keeps track of which
  degrees of freedom are related to which item, and how these degrees
  of freedom are mapped to the unknowns in the global system of
  equations. Degrees of freedom can be added and deleted on the fly,
  and all degrees of freedom related to an item will be automatically
  deleted whenever that item is deleted.

\item \Code{Globdat} -- a kind of common block on steroids for storing
  all `global' data in a program. The \Code{Globdat} class is actually
  a sort of mini namespace that provides a set of string constants and
  utility functions; the actual data are stored in a
  \Code{jem::util::Properties} object. The \Code{Globdat} class also
  defines a set of conventions that specify how the data are stored in
  a \Code{jem::\-util::\-Properties} object. Many classes in Jive will
  help you follow these conventions by providing member functions that
  store and retrieve instances of that class in and from a
  \Code{jem::\-util::\-Properties} object containing the global data.

\end{itemize}


%........................................................................

\subsection{New classes in the \Code{algebra} package}

\begin{itemize}

\item \Code{MatrixBuilder} -- provides a rich set of functions for
  building (sparse) matrices.

\item \Code{SparseMatrixBuilder} -- builds a sparse matrix of which
  the structure is known beforehand.

\item \Code{FlexMatrixBuilder} -- builds a sparse matrix with an
  arbitrary structure; that is, the structure is determined on the fly
  as matrix entries are set.

\item \Code{LumpedMatrixBuilder} -- builds a diagonal matrix by
  summing all off-diagonal matrix entries to the diagonal entry on
  the same row.

\end{itemize}


%........................................................................

\subsection{New classes in the \Code{app} package}

\begin{itemize}

\item \Code{Module} -- a basic block for building programs. A
  \Code{Module} object should concentrate on performing a specific,
  well-defined task so that it can be re-used in different programs.
  In the same way that Unix programs can be connected with pipes,
  \Code{Module} objects can be connected in a chain (see below); they
  communicate with each other through a
  \Code{jem::\-util::\-Properties} object containing the global data
  of the program.

\item \Code{ChainModule} -- a module that connects a set of modules in
  a chain; the output of one module will serve as the input of the
  next module in the chain.

\item \Code{CheckpointModule} -- a module that periodically dumps the
  entire state of a program (stored in a
  \Code{jem::\-util::\-Properties} object) to a binary file. This
  module also enables a program to continue from a previously saved
  state.

\item \Code{UserconfModule} -- a module that will create a hierarchy
  of modules at runtime given a description of this hierarchy in a
  configuration file. This module makes it possible to modify the
  top-level structure of a program after it has been compiled.

\item \Code{ModuleFactory} -- a factory class that creates new
  \Code{Module} objects given a set of configuration parameters. By
  using this class a program can create modules of which the
  type is specified by a user in a configuration file.

\end{itemize}


%........................................................................

\subsection{New classes in the \Code{geom} package}

\begin{itemize}

\item \Code{ParametricShape} -- an \Code{Internal\-Shape} class that
  implements a parametric mapping of a local coordinate system to the
  global coordinate system. You can let this class handle the local to
  global transformation of your own shape functions.

\item \Code{ParametricBoundary} -- a \Code{Boundary\-Shape} class that
  implements a parametric mapping for boundary shapes.

\item \Code{ShapeFactory} -- a factory class that creates new
  \Code{Shape} objects given a set of configuration parameters.

\item \Code{IShapeFactory} -- a factory class that creates
  \Code{InternalShape} objects.

\item \Code{BShapeFactory} -- a factory class that creates
  \Code{BoundaryShape} objects.

\end{itemize}


%........................................................................

\subsection{New classes in the \Code{model} package}

\begin{itemize}

\item \Code{Model} -- encapsulates a numerical model. The \Code{Model}
  class essentially has one member function -- named \Code{takeAction}
  -- that can be used to send `commands' to a model. This function has
  three arguments: a string that specifies what action is to be taken;
  a \Code{jem::\-util::\-Properties} object containing all input and
  output parameters for this specific action; and another
  \Code{jem::\-util::\-Properties} object containing all global data
  in a program. Because the interface of the \Code{Model} class is
  very generic, it enables one to combine independently developed
  models in one program, provided that they use the same conventions
  for storing the action-specific and global data.

\item \Code{DebugModel} -- a model that wraps around another model and
  that prints debugging information whenever it is sent a command
  through its \Code{takeAction} member function.

\item \Code{MultiModel} -- a model that groups a set of \Code{Model}
  objects into a single model. That is, a \Code{MultiModel} will
  forward each command to all models that it contains.

\item \Code{ModelFactory} -- a factory class for creating \Code{Model}
  objects given a set of configuration parameters.

\end{itemize}


%........................................................................

\subsection{New classes in the \Code{solver} package}

\begin{itemize}

\item \Code{Constraints} -- encapsulates a set of linear constraints.
  The \Code{Constraints} class provides functions for storing,
  retrieving and deleting both point constraints and linear
  constraints involving multiple degrees of freedom. Dependencies
  between constraints are automatically resolved as new constraints
  are added to a \Code{Constraints} object. A \Code{Constraints}
  object can also track a \Code{DofSpace} object and automatically
  reorder and delete constraints when the degrees of freedom in the
  \Code{DofSpace} object are reordered or deleted.

\item \Code{SolverFactory} -- a factory class that creates
  \Code{Solver} object given a set of configuration parameters.

\end{itemize}


%........................................................................

\subsection{New classes in the \Code{fem} package}

\begin{itemize}

\item \Code{NodeSet} -- an \Code{ItemsWrapper} class that wraps around
  a \Code{PointSet} and that provides more logical names for the
  member functions of the \Code{PointSet} class.

\item \Code{ElementSet} -- an \Code{ItemsWrapper} class that wraps
  around a \Code{GroupSet} in which each group represents an element.
  The members of a group are the nodes attached to an element. The
  \Code{Element\-Set} provides familiar names for the member functions
  of the \Code{GroupSet} class.

\item \Code{BoundarySet} -- an \Code{ItemsWrapper} class that wraps
  around a \Code{Member\-Set} in which each member represents an
  element boundary.

\item \Code{NodeGroup} -- an \Code{GroupWrapper} class that wraps
  around a \Code{ItemGroup} that selects a set of nodes.

\item \Code{ElementGroup} -- an \Code{GroupWrapper} class that wraps
  around a \Code{ItemGroup} that selects a set of elements.

\item \Code{ElementGroup} -- an \Code{GroupWrapper} class that wraps
  around a \Code{ItemGroup} that selects a set of element boundaries.

\item \Code{Actions} -- defines a set of standard actions that can be
  implemented by a finite element \Code{Model}. To be precise, the
  \Code{Actions} class defines a set of string constants that can be
  passed as the first argument to the \Code{take\-Action} member
  function of the \Code{Model} class.

\item \Code{Globdat} -- defines a set of string constants that can be
  used to access finite element data in a
  \Code{jem::\-util::\-Properties} object containing the global data
  in a program. Although you are free to give your data objects any
  names you like, using the names defined by the \Code{Globdat} will
  facilitate the use of both the classes in Jive and the software
  components that have been developed by others.

\item \Code{InputModule} -- a \Code{Module} class that reads input
  data -- including nodes, elements and tables -- from a file and
  stores those data in a \Code{jem::\-util::\-Properties} object
  containing the global data in a program.

\item \Code{InitModule} -- a \Code{Module} class that creates a
  \Code{Model} object and also initializes some other global data such
  as constraints.

\item \Code{InfoModule} -- a module that prints some information about
  the state and progress of the current program.

\item \Code{LinsolveModule} -- a \Code{Module} class that assembles a
  linear system of equations and computes the solution of that
  system. The system is assembled by sending a specific command to a
  \Code{Model} object.

\item \Code{OutputModule} -- a \Code{Module} class that writes various
  data sets to an output file. The data sets to be written and the
  name of the file can be specified in a configuration file.

\end{itemize}


%------------------------------------------------------------------------

\section{Porting from Jive~0.9}

Porting a program from Jive~0.9 to Jive~1.0 mostly involves replacing
obsolete classes and functions from Jive~0.9 by equivalent classes and
functions from Jive~1.0. The table below indicates which classes from
Jive~0.9 can be replaced by which classes from Jive~1.0.

The modifications that are required on a more structural level are
mainly limited to the way in which a global system of equations is
assembled. In Jive~0.9 you had to create a set of element locators
that had to be passed to a \Code{MatrixAssembler} and a
\Code{VectorAssembler}. In Jive~1.0 you do not have to explicitly
create the element locator anymore; instead, you can use a
\Code{DofSpace} object to find out how the entries in an element
matrix/vector should be mapped to the corresponding entries in the
global matrix/vector. This design is not only more intuitive in its
use, but also makes it possible to assemble a matrix of which the
structure is not known beforehand.


\renewcommand{\arraystretch}{1.4}

\setlongtables

\vspace{4ex}

\begin{longtable}[c]{|p{7cm}|>{\PBS\raggedright}p{7cm}|}
  \hline
  \textbf{Obsolete class in Jive~0.9} &
  \textbf{Equivalent class(es) in Jive~1.0} \\
  \hline
  \endhead
  \hline
  \textbf{Obsolete class in Jive~0.9} &
  \textbf{Equivalent class(es) in Jive~1.0} \\
  \hline \hline
  \endfirsthead
  \endfoot
  \hline
  \endlastfoot
  %
  \Code{util::Mesh} &
  \Code{fem::NodeSet}, \Code{fem::ElementSet} \\
  \hline
  \Code{util::Mesh\-Boundary} &
  \Code{fem::BoundarySet} \\
  \hline
  \Code{util::Mesh\-Parser} &
  \Code{util::DataParser} or \Code{fem::DataParser} \\
  \hline
  \Code{util::Index\-Space} &
  \Code{util::ItemSet} \\
  \hline
  \Code{util::IndexSet} &
  \Code{util::ItemGroup} \\
  \hline
  \Code{util::Locator\-Utils} &
  \Code{util::DofSpace} \\
  \hline \hline
  \Code{algebra::StdSparse\-Matrix} &
  \Code{algebra::Sparse\-Matrix} \\
  \hline \hline
  \Code{solver::Point\-Constraints} &
  \Code{solver::Constraints} \\
  \hline
  \Code{solver::Linear\-Constraints} &
  \Code{solver::Constraints} \\
  \hline
  \Code{solver::StdPoint\-Constrainer} &
  \Code{solver::Point\-Constrainer} \\
  \hline
  \Code{solver::Direct\-Point\-Constrainer} &
  \Code{solver::Point\-Constrainer} \\
  \hline
  \Code{solver::Linear\-Constrainer} &
  \Code{solver::StdConstrainer} \\
  \hline \hline
  \Code{fem::Matrix\-Assembler} &
  \Code{algebra::Matrix\-Builder} \\
  \hline
  \Code{fem::Sparse\-Matrix\-Assembler} &
  \Code{algebra::Flex\-Matrix\-Assembler} or
  \Code{fem::FEMatrix\-Assembler} \\
  \hline
  \Code{fem::Vector\-Assembler} &
  no longer needed; use class \Code{util::DofSpace} and the function
  \Code{jem::select} \\
  \hline
  \Code{fem::Element} &
  \Code{model::Model} \\
  \hline
  \Code{fem::Boundary\-Element} &
  \Code{model::Model} \\
\end{longtable}


\end{document}
